\documentclass[12pt]{amsart}

% theorem style
\newtheorem{thm}{Theorem}[section]
\newtheorem{cor}[thm]{Corollary}
\newtheorem{lem}[thm]{Lemma}
\newtheorem{prop}[thm]{Proposition}
\newtheorem{prob}[thm]{Problem}

% non-theorem style
\theoremstyle{definition}
\newtheorem{defin}[thm]{Definition}
\newtheorem{rem}[thm]{Remark}
\newtheorem{exa}[thm]{Example}

% macros go here

\begin{document}

\title[]{Proof of the Quadratic Equation}
\author{Justin Hood}
\address{}
\email{hoodj5402@uwstout.edu}

%\date{September 7, 2018}

\begin{abstract}
In the paper that follows, an analytical proof of the so called ``quadratic formula" shall be derived.  Implementing the method of completing the square \cite{complete}, and some clever algebraic manipulation, we will derive the classic formula used to compute roots of quadratic equations.
\end{abstract}

\maketitle

\section{Quadratic Theorem}
A quadratic equation in mathematics is a polynomial equation wherein the highest order power of the unknown variable is $2$. Thus, we note that in general, a quadratic equation takes the form, $ax^2+bx+c=0$, with $a\neq 0$. With this definition in mind, we consider the following theorem.
\thm[Quadratic Theorem]\hfill\\\\
Let an equation of the form,
\begin{equation}\label{quad}
Ax^2+Bx+C=0
\end{equation}
exist, with $A,B,C\in\mathbb{C}$, $A\neq 0$. Then, the solutions of this equation shall take the form,
\[x=\frac{-B\pm\sqrt{B^2-4AC}}{2A}\]
\hfill\\
\section{Proof}
\begin{proof}
Let $A,B,C$ exist, such that $A,B,C\in\mathbb{C}$, and $A\neq 0$. Then, we consider equation (\ref{quad}), and perform some basic algebraic manipulation.
\begin{align*}
Ax^2+Bx+C & =0 &&\\
x^2+\frac{B}{A}x+\frac{C}{A}&=0 && \text{Divide by $A$} \\
x^2+\frac{B}{A}x&=-\frac{C}{A} && \text{Subtract the constant} \\
\end{align*}
Next, we shall ``Complete the Square"\cite{complete}, by introducing a new constant to both sides, and manipulating the LHS of the equation into the square of a binomial,
\begin{align*}
x^2+\frac{B}{A}x+\bigg(\frac{B}{2A}\bigg)^2&=-\frac{C}{A}+\bigg(\frac{B}{2A}\bigg)^2 && \text{Add the new constant} \\
\bigg(x+\frac{B}{2A}\bigg)^2 &= \bigg(\frac{B}{2A}\bigg)^2-\frac{C}{A} && \text{Factor into the binomial form}
\end{align*}
Finally, we may begin to solve for $x$,
\begin{align*}
\bigg(x+\frac{B}{2A}\bigg)^2 &= \frac{B^2-4AC}{4A^2} && \text{Simplify RHS}\\
x+\frac{B}{2A}&=\pm\sqrt{\frac{B^2-4AC}{4A^2}} && \text{Take the square root} \\
x&=-\frac{B}{2A}\pm\frac{\sqrt{B^2-4AC}}{2A} && \text{Simplify}\\
x&=\frac{-B\pm\sqrt{B^2-4AC}}{2A}
\end{align*}
Thus, we have arrived at our desired result, and shown that the solutions of a quadratic equation adhere to the ``Quadratic Formula".
\end{proof}
\newpage
\begin{thebibliography}{9}

\bibitem{complete}
Stapel, E.
\textit{Completing the Square: Solving Quadratic Equations | Purplemath}. [online] Purplemath. Available at: 
\texttt{https://www.purplemath.com/modules/sqrquad.htm}

\end{thebibliography}

\end{document}
